\subsection{Partitioning Algorithms}
There are multiple partitioning algorithms in use and in study currently.  Not all of the algorithms have the same goal.  Hash-based algorithms help scale the database by evenly distributing the data on different nodes while other algorithms have a more specific goal to reduce transaction overhead.

Schism ~\cite{Curino:2010:SWA:1920841.1920853} is a static partitioning algorithm to reduce distributed transactions for SQL datastores. It uses transaction logs to determine how to partition data. The transaction analysis is similar to the work done by Cheng et al.~\cite{Cheng:2002} It significantly improves performance compared to hash-based and even manual partitioning techniques. These static algorithms are a good starting point for our work and would likely yield better results for key/value stores given their simplified data access model. Also, an incremental version of these algorithms may scale better than the static equivalent, since it may be able to consider only new transaction log entries in each iteration. Another option would be to use a probabilistic algorithm.

Hehme and Bruno ~\cite{Nehme:2011:APD:1989323.1989444} present and algorithm that deeply integrates directly with the parallel query optimizer in Microsoft SQL Server.  Their algorithm provides a static data partitioning recommendation.  Their goal was to provide a data partition strategy in less time than other less deeply integrated solutions.
